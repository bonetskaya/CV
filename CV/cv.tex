\documentclass{article}

\usepackage{cmap}
\usepackage[T2A]{fontenc}
\usepackage[utf8]{inputenc}
\usepackage[english,russian]{babel}
\usepackage{indentfirst}
\usepackage{amssymb}
\usepackage{amsmath}
\usepackage{multicol}

\usepackage{geometry}
\geometry{top=20mm}
\geometry{bottom=10mm}
\geometry{left=20mm}
\geometry{right=10mm}

\pagenumbering{gobble}

\usepackage[dvipsnames]{xcolor}
\usepackage[colorlinks  = true,
            linkcolor   = black,
            urlcolor    = black,
            citecolor   = black,
            anchorcolor = black]{hyperref}

\renewcommand{\maketitle}{
    \begin{center}
        \Huge
        \textbf{Ольга Бонецкая}
    \end{center}

    \footnotesize
    \begin{minipage}{0.4\textwidth}
        Телефон: $\boldsymbol{+}$7(925)458-20-91\\[0.3em]
        Email: \href{mailto:oabonetskaya@edu.hse.ru}{oabonetskaya@edu.hse.ru}
    \end{minipage}
    \hfill
    \begin{minipage}{0.4\textwidth}
        \begin{flushright}
            Telegram: {@oly\_bo}\\[0.3em]
            GitHub: \href{https://github.com/olybo}{olybo}
        \end{flushright}
    \end{minipage}
}

\usepackage{titlesec}
\titleformat{\section}{\Large\bf\raggedright}{}{0.5em}{}[{\titlerule[1pt]}]
\titlespacing{\section}{0pt}{3pt}{7pt}
\titleformat{\subsection}{\large\bfseries\raggedright}{}{0em}{\underline}%[\rule{3cm}{.2pt}]
\titlespacing{\subsection}{0pt}{7pt}{7pt}


\newcommand{\entry}[3]{
    \begin{minipage}[t]{.11\linewidth}
        \hfill \textsc{#1}
    \end{minipage}
    \hfill\vline\hfill
    \begin{minipage}[t]{.80\linewidth}
        \textbf{#2}\\
        \footnotesize{#3}
    \end{minipage}
}

\begin{document}
    \maketitle

    \section{Образование}
        \entry {2017 -- 2021}
        {Национальный исследовательский университет «Высшая школа экономики»,\\
         Факультет компьютерных наук, Прикладная математика и информатика}
        {Очная форма обучения, бакалавриат, 3 курс, программа <<Распределённые системы>>}


    \section{Курсы}
        \entry {2019}
        {Машинное обучение-1,\\Факультет компьютерных наук НИУ ВШЭ}
        {}
        
        \vspace{.2cm}
        \entry {2020}
        {Машинное обучение-2,\\Факультет компьютерных наук НИУ ВШЭ}
        {В процессе.}
        
        \vspace{.2cm}
        \entry {2020}
        {Advanced Machine Learning Specialization,\\National Research University Higher School of Economics}
        {Онлайн-курс на платформе Coursera (в процессе).}
        
        \section{Проекты}
        \entry {2018-2019}
        {Программный проект по теме "Кластеризация сложных сетей"}
        {Часть группового проекта по исследованию сложных сетей, тестировала различные алгоритмы кластеризации. Язык Python3.}
        
        

    \section{Опыт работы}
    \entry {2017 -- 2020} 
    {Преподаватель в Лицее НИУ ВШЭ}
    {Вела курсы <<Практикум по программированию>> на языке C++ для 10 и 11 классов. Читала курс <<Олимпиадного программирования>> для 9-11 классов.}

    \vspace{.2cm}
    \entry {2020} 
    {Ассистент онлайн-курса НИУ ВШЭ <<Основы машинного обучения>>}
    {Занималась подготовкой и созданием методических материалов для курса, помогала слушателям курса в освоении материала.}


    \section{Навыки}
    \begin{multicols}{2}
        \subsection{Теоретические знания}
        \begin{itemize}
                \item Алгоритмы и структуры данных
                \begin{itemize}
                        \item алгоритмы на графах
                        \item динамическое программирование
                        \item структуры данных
                \end{itemize}
                \item Алгоритмы машинного обучения
                \begin{itemize}
                        \item Регрессия (линейная, решающие деревья)
                        \item Классификация (линрег, SVM, деревья решений)
                        \item Кластеризация (K-means, DBScan)
                        \item Ансабли (бэггинг, бустинг)
                        \item Перцептрон (полносвязные, свёрточные сети)
                \end{itemize}
                \item Математическая статистика
                \item Линейная алгебра, математический анализ
                \item Дискретная математика
            \end{itemize}
        \subsection{Python3}
            \begin{itemize}
                \item Математика (numpy)
                \item Алгоритмы машинного обучения (sklearn, Catboost)
                \item Визуализация (pandas, matplotlib, seaborn)
                \item Глубинное обучение (keras, tensorflow)
            \end{itemize}
            \subsection{Что-то ещё}
            \begin{itemize}
                \item Google Colab
                \item Jupyter Notebook
                \item \LaTeX{}
                \item Linux (Ubuntu, Alpine)
                \item C/C++, Go
                \item SQL (Postgres)
                \item Docker, docker-compose
                \item git
            \end{itemize}

    \end{multicols}
    \section{Другое}
        \begin{itemize}
            \item Английский Upper Intermediate, читаю техническую литературу
        \end{itemize}


\end{document}
